\section{Backporting strategies}
\label{strategies}

We now review three backporting strategies: i) the common strategy of
providing version-specific implementations using \#ifdefs, ii) the strategy
originally taken by the Linux backports project of factorizing these
changes into a compatibility library, and iii) our extension of the latter
using Coccinelle.

\subsection{Backporting with \#ifdefs}

Backporting a device driver typically consists of modifying the source code
with \#ifdefs to handle the different requirements of different kernel
releases.  This entails adding blocks of code that provide alternate
implementations for various functionalities, for different ranges of kernel
versions, according to which evolutions have occurred and which collateral
evolutions must be performed to accommodate them.

As a running example, we consider an evolution that was introduced by Linux
kernel commit d314774cf2 and that was first merged upstream in Linux
v2.6.29.  This evolution moved a series of callback functions from the {\tt
  net\_\-device} data structure out into a new separate data structure of
type {\tt net\_\-device\_\-ops}.  Backporting over this evolution, for Linux
kernel versions before Linux v2.6.29, requires collateral evolutions that
{\em undo} this change.  Specifically, the definition of the new {\tt
  net\_\-device\_\-ops} structure and the initialization of the link from the
{\tt net\_\-device} structure to this new structure must be restricted, using
an \#ifdef, to the versions starting with Linux v2.6.29.  Earlier versions
must initialize the appropriate fields in the {\tt net\_\-device} structure
itself, from among the callback functions that the modern driver puts in the
new structure.  Figure \ref{usbport} shows a patch that backports this
single collateral evolution on one device driver.

\begin{figure}
\begin{lstlisting}[language=diff]
--- a/drivers/net/usb/usbnet.c
+++ b/drivers/net/usb/usbnet.c
@@ -1151,6 +1151,7 @@
 }
 EXPORT_SYMBOL_GPL(usbnet_disconnect);
 
+#if (LINUX_VERSION_CODE >= KERNEL_VERSION(2,6,29))
 static const struct net_device_ops usbnet_netdev_ops = {
   .ndo_open               = usbnet_open,
   .ndo_stop               = usbnet_stop,
@@ -1160,6 +1161,7 @@
   .ndo_set_mac_address    = eth_mac_addr,
   .ndo_validate_addr      = eth_validate_addr,
 };
+#endif
 
 /*-----------------------------------------------------*/
 
@@ -1229,7 +1231,15 @@
     net->features |= NETIF_F_HIGHDMA;
 #endif
 
+#if (LINUX_VERSION_CODE >= KERNEL_VERSION(2,6,29))
   net->netdev_ops = &usbnet_netdev_ops;
+#else
+  net->change_mtu = usbnet_change_mtu;
+  net->hard_start_xmit = usbnet_start_xmit;
+  net->open = usbnet_open;
+  net->stop = usbnet_stop;
+  net->tx_timeout = usbnet_tx_timeout;
+#endif
   net->watchdog_timeo = TX_TIMEOUT_JIFFIES;
   net->ethtool_ops = &usbnet_ethtool_ops;
\end{lstlisting}
\caption{Backporting the {\tt net\_\-device\_\-ops} collateral evolution for
  the {\tt usbnet} driver}
\label{usbport}
\end{figure}

Lines 7 and 23 of Figure \ref{usbport} introduce the \#ifdefs that restrict
some code of the modern driver to be used only in Linux versions v2.6.29
and later.  Lines 25-31 introduce the code to be used for earlier versions,
placing the relevant callback functions from the modern code (lines 8-13)
into the fields of the {\tt net} structure.  In each case, the callback
functions are used by the driver support library of the kernel, which is
not backported, and which thus finds the desired functions in the expected
place, with no further code changes.

In general, every driver that initializes a {\tt net\_\-device} structure
would require all of these changes.  Creating these patches, and
maintaining them as other collateral evolutions are needed, is complex,
tedious, and error prone.
%\jl{Who uses this approach?}

\subsection{Backports via a compatibility library}

Maintenance of patches is easy as long as the amount of changes being
introduced is rather small. The {\tt netdev\_\-ops} collateral evolution,
however, is an example of a collateral evolution that affects many network
drivers, resulting in a large set of changes, that then have to be
maintained in patch form.  A better approach, proposed by the Linux
backports project, consists of wrapping up the required changes into static
inline or external helper functions and then using \#ifdefs in these
functions to adapt the code to each previous release.

This strategy is illustrated by the following code, which backports the
{\tt netdev\_\-ops} collateral evolution for two device drivers.  Now, the
new {\tt net\_\-device\_\-ops} structure used by the modern driver is
maintained as is.  Instead, we replace the direct initialization of the
{\tt netdev\_\-ops} field by a call to a single static inline function
defined by the backports compat library, amounting to glue code.  Now, only
this function needs multiple lines of \verb-#ifdef- code, performing the
direct assignment for versions starting with Linux v2.6.29, and copying the
fields from the new structure into the main {\tt net\_\-device} structure
for the older versions.  Only one line of code is changed in each driver,
in contrast to the 10 lines added to each driver by the previous approach.

\begin{lstlisting}[language=diff]
--- a/drivers/net/usb/usbnet.c
+++ b/drivers/net/usb/usbnet.c
@@ -1446,7 +1446,7 @@ usbnet_probe (struct usb_interface *udev
     net->features |= NETIF_F_HIGHDMA;
 #endif
 
-  net->netdev_ops = &usbnet_netdev_ops;
+  netdev_attach_ops(net, &usbnet_netdev_ops);
   net->watchdog_timeo = TX_TIMEOUT_JIFFIES;
   net->ethtool_ops = &usbnet_ethtool_ops;
 
--- a/drivers/net/wireless/ath/ath6kl/main.c
+++ b/drivers/net/wireless/ath/ath6kl/main.c
@@ -1289,7 +1289,7 @@ static const struct net_device_ops ath6k
 
 void init_netdev(struct net_device *dev)
 {
-  dev->netdev_ops = &ath6kl_netdev_ops;
+  netdev_attach_ops(dev, &ath6kl_netdev_ops);
   dev->destructor = free_netdev;
   dev->watchdog_timeo = ATH6KL_TX_TIMEOUT;
\end{lstlisting}

Between 2007 and 2013 the backports project exclusively followed this
strategy to help reduce the amount of maintenance on patches. The backport
compat library now has a large set of helper functions that help to keep the
number and size of the patches required for each backported driver to a
minimum.

\subsection{The Coccinelle way to backport}

The compat library approach reduces significantly the amount of code that
must be modified in each driver. Still, backporting a new driver requires
identifying the set of features that it uses, and comparing these features
to those provided by the compat library to see where a collateral evolution
to replace the existing code by glue code invoking the compat library is
needed.  Just as Coccinelle had been found to be useful in performing
traditional (forward) collateral evolutions, we considered whether
Coccinelle could also be useful for the kinds of collateral evolutions
required in backports. For example, the {\tt netdev\_\-ops} collateral
evolution could be expressed as a Coccinelle semantic patch as follows:

\begin{lstlisting}[language=diff]
@@
struct net_device *dev;
struct net_device_ops ops;
@@
- dev->netdev_ops = &ops;
+ netdev_attach_ops(dev, &ops);
\end{lstlisting}

This semantic patch could be used to backport the netdev\_\-ops collateral
evolution for all networking device drivers.  It is indeed no longer even
necessary for the developer to identify whether a new device driver to
backport uses this features; Coccinelle both finds and updates the relevant
code automatically.  Note in particular that the semantic patch specifies
the type of the expressions matching the {\tt dev} and {\tt ops}
metavariables, to ensure that the transformation is performed only on
structures of the appropriate type.  Finally, this semantic patch amounts to
only 6 lines of code to maintain, rather than 2 lines of code {\em for each
  driver} with the \#ifdef approach.
