\documentclass[a4paper,10pt]{article}
\usepackage{color}
\usepackage[svgnames]{xcolor}
  \definecolor{diffstart}{named}{Grey}
  \definecolor{diffincl}{named}{Green}
  \definecolor{diffrem}{named}{Red}

\usepackage{listings}
  \lstdefinelanguage{diff}{
	basicstyle=\ttfamily\scriptsize,
	morecomment=[f][\color{diffstart}]{@@},
	morecomment=[f][\color{diffincl}]{+},
	morecomment=[f][\color{diffrem}]{-},
	identifierstyle=\color{black},
  }

\lstloadlanguages{Bash}
\lstset{language=Bash,
	basicstyle=\ttfamily\scriptsize,
        showspaces=false,
        rulesepcolor=\color{gray},
	showstringspaces=false,
	keywordstyle=\bfseries\color{green!40!black},
	commentstyle=\itshape\color{purple!40!black},
	identifierstyle=\color{blue},
	stringstyle=\color{red},
        morekeywords={elif}
       }

\lstloadlanguages{C}
\lstset{language=C,
	basicstyle=\ttfamily\scriptsize,
	backgroundcolor=\color{white},
        showspaces=false,
        rulesepcolor=\color{gray},
	showstringspaces=false,
	keywordstyle=\bfseries\color{blue!40!black},
	commentstyle=\itshape\color{purple!40!black},
	identifierstyle=\color{black},
	stringstyle=\color{red},
        morekeywords={elif}
       }

\usepackage{caption}
\usepackage{hyperref}
\DeclareCaptionFont{white}{\color{white}}
\DeclareCaptionFormat{listing}{\colorbox{gray}{\parbox{\textwidth}{#1#2#3}}}
\captionsetup[lstlisting]{format=listing,labelfont=white,textfont=white}

\begin{document}

\begin{titlepage}
\begin{center}

\textsc{Backporting the Linux kernel with SmPL}\\[1.0cm]

\begin{minipage}{0.4\textwidth}
\begin{flushleft} \large
Luis R. Rodriguez mcgrof@suse.com
\end{flushleft}
\end{minipage}

\vfill

{\large \today}

\end{center}
\end{titlepage}

\section{Backporting the Linux kernel with SmPL}

We should not only consider how to better evolve software but also how to more
efficiently backport it. The Coccinelle engine was originally developed to help
evolve writing software faster and more efficiently, but it can also be used to
help backport software. This paper will cover how SmPL was embraced by the
Linux backports project to help it backport the Linux kernel more effectively,
more automatically, and to help the effort scale.

\subsection{A brief on the Linux kernel backports Project}

\begin{enumerate}
\item
Project home page: \url{https://backports.wiki.kernel.org}
\item
Code: \url{https://git.kernel.org/cgit/linux/kernel/git/backports/backports.git/}
\item
IRC: irc.freenode.net \#kernel-backports
\item
Dev team: 3 core developers, 2 co-maintainers, Hauke Mehrtens now doing most of the work
\item
Subsystems backported: Ethernet, Wireless, Bluetooth, NFC, ieee802154, Media, Regulator
\end{enumerate}

The Linux kernel backports project started in 2007, it has undergone
a series of project names but eventually it took the general backports
name as the project started backporting many subsystems. The project
eventually folded under the Linux Foundation backports working group
and now spearheads that effort.

The backports project strives to backport the Linux kernel automatically
through a series of strategies. It provides stable RC and first point
stable releases based on Linus' tree and extra version stable releases
based Greg's linux-stable tree. The method of development consists
of following linux-next and making snapshot releases based on that. During the
merge window one snapshot is used to base a branch off for a stable
release. It follows the same mnemonic as used by linux-stable for stable
release branches, linux-3.17.y, linux-3.16.y, while the master branch always
follows linux-next.

The project backports well over 800 drivers now, originally it provided
backporting support for drivers down to 2.6.25. In order to scale better
however the project now only supports kernels down to up to the the 3.x
kernels, it strives to at least support kernels listed on kernel.org.
The original motivation behind the project was to encourage
silicon manufacturers to work upstream on the Linux kernel while providing a
solution for them for backporting their drivers automatically down to older
releases.  The framework is designed only for Linux upstream drivers,
proprietary drivers cannot and should not use this framework.

\subsection{A slightly different use case of Coccinelle}

The Coccinelle engine was developed to help evolve the Linux kernel,
INRIA / IRILL help evolve and maintain it. The Coccinelle engine was
built after evaluating how the Linux kernel evolves, it originally
tried to address enabling Linux kernel developers make collateral
evolutions faster. It accomplished this by defining a language to
express collateral evolutions, the Semantic Patch Language (SmPL).
Developers write SmPL rules and use Coccinelle to apply the SmPL
rules / patches onto a target directory / file. A typical use case
for kernel developers is for them to use Coccinelle once or twice a
month to help create collateral evolution on a series of device
drivers or subsystems. The use case for backports is very different.
Coccinelle would be used daily and all written SmPL rules would be
applied against all supported subsystems and drivers.

\subsection{The old way of doing backporting}

Backporting device drivers / features typically consists of modifying source
code with \#ifdefs to handle different requirements for different kernel
releases. Below is an example which shows how developers would typically
backport a collateral evolution with \#ifdefs. The example backports the
collateral evolution introduced by commit d314774cf2 merged upstream as of the
v2.6.29 release which moved a series of callbacks from struct net\_device out
into its own data structure.  This addresses backporting this single collateral
evolution on one device driver.

\begin{lstlisting}[language=diff]
--- a/drivers/net/usb/usbnet.c
+++ b/drivers/net/usb/usbnet.c
@@ -1151,6 +1151,7 @@
 }
 EXPORT_SYMBOL_GPL(usbnet_disconnect);
 
+#if (LINUX_VERSION_CODE >= KERNEL_VERSION(2,6,29))
 static const struct net_device_ops usbnet_netdev_ops = {
 	.ndo_open		= usbnet_open,
 	.ndo_stop		= usbnet_stop,
@@ -1160,6 +1161,7 @@
 	.ndo_set_mac_address 	= eth_mac_addr,
 	.ndo_validate_addr	= eth_validate_addr,
 };
+#endif
 
 /*-------------------------------------------------------------------------*/
 
@@ -1229,7 +1231,15 @@
 		net->features |= NETIF_F_HIGHDMA;
 #endif
 
+#if (LINUX_VERSION_CODE >= KERNEL_VERSION(2,6,29))
 	net->netdev_ops = &usbnet_netdev_ops;
+#else
+	net->change_mtu = usbnet_change_mtu;
+	net->hard_start_xmit = usbnet_start_xmit;
+	net->open = usbnet_open;
+	net->stop = usbnet_stop;
+	net->tx_timeout = usbnet_tx_timeout;
+#endif
 	net->watchdog_timeo = TX_TIMEOUT_JIFFIES;
 	net->ethtool_ops = &usbnet_ethtool_ops;
\end{lstlisting}

\subsection{A slightly better way to backport}

Maintenance of patches is easy so long as the amount of changes being
introduced is rather small. The netdev\_ops collateral evolution is a good
example of a collateral evolution which would typically require a large series
of changes to maintain in patch form for each backported network device driver.
A slightly better approach consists of wrapping up the required changes into
static inline helpers or new exported symbols and have helper code do most of
the required work. Below is an example which reflects this strategy by
addressing backporting the netdev\_ops collateral evolution for two device
drivers. For years the backports project has been following this strategy to
help reduce the amount of maintenance on patches, it now has a large list of
static inlines and exported symbols which helps keep required patches deltas to
a minimum.

\begin{lstlisting}[language=diff]
--- a/drivers/net/usb/usbnet.c
+++ b/drivers/net/usb/usbnet.c
@@ -1446,7 +1446,7 @@ usbnet_probe (struct usb_interface *udev
 		net->features |= NETIF_F_HIGHDMA;
 #endif
 
-	net->netdev_ops = &usbnet_netdev_ops;
+	netdev_attach_ops(net, &usbnet_netdev_ops);
 	net->watchdog_timeo = TX_TIMEOUT_JIFFIES;
 	net->ethtool_ops = &usbnet_ethtool_ops;
 
--- a/drivers/net/wireless/ath/ath6kl/main.c
+++ b/drivers/net/wireless/ath/ath6kl/main.c
@@ -1289,7 +1289,7 @@ static const struct net_device_ops ath6k
 
 void init_netdev(struct net_device *dev)
 {
-	dev->netdev_ops = &ath6kl_netdev_ops;
+	netdev_attach_ops(dev, &ath6kl_netdev_ops);
 	dev->destructor = free_netdev;
 	dev->watchdog_timeo = ATH6KL_TX_TIMEOUT;
\end{lstlisting}
 

\subsection{The Coccinelle SmPL way to backport}

Just as the Linux kernel could be evolved with SmPL the idea of backporting
with SmPL was considered and evaluated. The netdev\_ops collateral evolution
could be expressed in SmPL grammar as follows, for example.

\begin{lstlisting}[language=diff]
@@
struct net_device *dev;
struct net_device_ops ops;
@@
-dev->netdev_ops = &ops;
+netdev_attach_ops(dev, &ops);
\end{lstlisting}

This grammar could be used to backport the netdev\_ops collateral evolution
for all networking device drivers.

\subsection{Testing the limits of Coccinelle, backporting complex changes}

To what extent can Coccinelle SmPL grammar be used to help backporting?
To test the limits on what could be backported the most complex collateral
evolution supported by the backports project was chosen as a test case,
it was decided to try to backport threaded IRQ support, introduced in the
v2.6.31 kernel.

\subsection{Backporting threaded IRQ support the old way}

Its important to first explain how the backports project provided backport
support for threaded IRQ support prior to trying to use SmPL for the backport.
Threaded IRQ support was supported on older kernels by extending each private
driver data structure that used request\_threaded\_irq() with a struct
compat\_threaded\_irq data structure defined as follows:

\begin{lstlisting}[language=C]
#if LINUX_VERSION_CODE < KERNEL_VERSION(2,6,31)
struct compat_threaded_irq {
	unsigned int irq;
	irq_handler_t handler;
	irq_handler_t thread_fn;
	void *dev_id;
	char wq_name[64];
	struct workqueue_struct *wq;
	struct work_struct work;
};
#endif
\end{lstlisting}

The following helpers are also defined as static inlines.

\begin{lstlisting}[language=C]
#if LINUX_VERSION_CODE < KERNEL_VERSION(2,6,31)
static inline
void compat_irq_work(struct work_struct *work)
{
	struct compat_threaded_irq *comp = container_of(work, struct compat_threaded_irq, work);
	comp->thread_fn(comp->irq, comp->dev_id);
}

static inline
irqreturn_t compat_irq_dispatcher(int irq, void *dev_id)
{
	struct compat_threaded_irq *comp = dev_id;
	irqreturn_t res;

	res = comp->handler(irq, comp->dev_id);
	if (res == IRQ_WAKE_THREAD) {
		queue_work(comp->wq, &comp->work);
		res = IRQ_HANDLED;
	}

	return res;
}

static inline
int compat_request_threaded_irq(struct compat_threaded_irq *comp,
				unsigned int irq,
				irq_handler_t handler,
				irq_handler_t thread_fn,
				unsigned long flags,
				const char *name,
				void *dev_id)
{
	comp->irq = irq;
	comp->handler = handler;
	comp->thread_fn = thread_fn;
	comp->dev_id = dev_id;
	INIT_WORK(&comp->work, compat_irq_work);

	if (!comp->wq) {
		snprintf(comp->wq_name, sizeof(comp->wq_name),
			 "compirq/%u-%s", irq, name);
		comp->wq = create_singlethread_workqueue(comp->wq_name);
		if (!comp->wq) {
			printk(KERN_ERR "Failed to create compat-threaded-IRQ workqueue %s\n",
			       comp->wq_name);
			return -ENOMEM;
		}
	}
	return request_irq(irq, compat_irq_dispatcher, flags, name, comp);
}

static inline
void compat_free_threaded_irq(struct compat_threaded_irq *comp)
{
	free_irq(comp->irq, comp);
}

static inline
void compat_destroy_threaded_irq(struct compat_threaded_irq *comp)
{
	if (comp->wq)
		destroy_workqueue(comp->wq);
	comp->wq = NULL;
}

static inline
void compat_synchronize_threaded_irq(struct compat_threaded_irq *comp)
{
	synchronize_irq(comp->irq);
	cancel_work_sync(&comp->work);
}
#endif
\end{lstlisting}

Drivers would then need to be modified in order to make use of the new
helpers. For example threaded IRQ support was supported for the b43 device driver
through the following two patches.

\begin{lstlisting}[language=diff]
--- a/drivers/net/wireless/b43/b43.h
+++ b/drivers/net/wireless/b43/b43.h
@@ -805,6 +805,9 @@ enum {
 
 /* Data structure for one wireless device (802.11 core) */
 struct b43_wldev {
+#if LINUX_VERSION_CODE < KERNEL_VERSION(2,6,31)
+	struct compat_threaded_irq irq_compat;
+#endif
 	struct b43_bus_dev *dev;
 	struct b43_wl *wl;
 	/* a completion event structure needed if this call is asynchronous */
--- a/drivers/net/wireless/b43/main.c
+++ b/drivers/net/wireless/b43/main.c
@@ -4243,8 +4243,17 @@ redo:
 	if (b43_bus_host_is_sdio(dev->dev)) {
 		b43_sdio_free_irq(dev);
 	} else {
+#if LINUX_VERSION_CODE >= KERNEL_VERSION(2,6,31)
 		synchronize_irq(dev->dev->irq);
+#else
+		compat_synchronize_threaded_irq(&dev->irq_compat);
+#endif
+#if LINUX_VERSION_CODE >= KERNEL_VERSION(2,6,31)
 		free_irq(dev->dev->irq, dev);
+#else
+		compat_free_threaded_irq(&dev->irq_compat);
+		compat_destroy_threaded_irq(&dev->irq_compat);
+#endif
 	}
 	mutex_lock(&wl->mutex);
 	dev = wl->current_dev;
@@ -4290,9 +4299,17 @@ static int b43_wireless_core_start(struc
 			goto out;
 		}
 	} else {
+#if LINUX_VERSION_CODE >= KERNEL_VERSION(2,6,31)
 		err = request_threaded_irq(dev->dev->irq, b43_interrupt_handler,
 					   b43_interrupt_thread_handler,
 					   IRQF_SHARED, KBUILD_MODNAME, dev);
+#else
+		err = compat_request_threaded_irq(&dev->irq_compat,
+						  dev->dev->irq,
+						  b43_interrupt_handler,
+						  b43_interrupt_thread_handler,
+						  IRQF_SHARED, KBUILD_MODNAME, dev);
+#endif
 		if (err) {
 			b43err(dev->wl, "Cannot request IRQ-%d\n",
 			       dev->dev->irq);
\end{lstlisting}

\subsection{Backporting threaded IRQ support with SmPL}

The challenge at hand is to now backport this threaded IRQ support by using
SmPL. Most changes ended up being relatively trivial -- replace one caller with
another caller, and instead of using one data structure, use the backport data
structure on some arguments of the callers.  The more challenging aspect of
this backport consisted of having to use Coccinelle to let it infer what the
private data structure was, and modify that to extend it with the backport IRQ
data structure. Coccinelle supports data structure inferences so after
understanding how to do that the backport was completed.

\begin{lstlisting}[language=diff]
@ threaded_irq @
identifier ret;
expression irq, irq_handler, irq_thread_handler, flags, name;
type T;
T *private;
@@

+#if LINUX_VERSION_CODE >= KERNEL_VERSION(2,6,31)
ret = request_threaded_irq(irq,
			   irq_handler,
			   irq_thread_handler,
			   flags,
			   name,
			   private);
+#else
+ret = compat_request_threaded_irq(&private->irq_compat,
+				   irq,
+				   irq_handler,
+				   irq_thread_handler,
+				   flags,
+				   name,
+				   private);
+#endif

@ sync_irq depends on threaded_irq @
expression irq;
type threaded_irq.T;
T *threaded_irq.private;
@@

+#if LINUX_VERSION_CODE >= KERNEL_VERSION(2,6,31)
synchronize_irq(irq);
+#else
+compat_synchronize_threaded_irq(&private->irq_compat);
+#endif

@ free depends on threaded_irq @
expression irq, dev;
type threaded_irq.T;
T *threaded_irq.private;
@@

+#if LINUX_VERSION_CODE >= KERNEL_VERSION(2,6,31)
free_irq(irq, dev);
+#else
+compat_free_threaded_irq(&private->irq_compat);
+compat_destroy_threaded_irq(&dev->irq_compat);
+#endif

@ modify_private_header depends on threaded_irq @
type threaded_irq.T;
@@

T {
+#if LINUX_VERSION_CODE < KERNEL_VERSION(2,6,31)
+      struct compat_threaded_irq irq_compat;
+#endif
...
};
\end{lstlisting}

\subsection{Legacy patch series to SmPL equivalence proof}

Coccinelle would typically be used to help yield a series of patches which
would then be used to be submitted for review upstream on the Linux kernel. For
backporting however we want to reduce into grammar what a series of patches
were previously doing. To verify correctness then one would need proof that a
new series of SmPL rules have a direct functional equivalence to an old series
of patches. Since patch formats vary a consistent patch format would need to be
decided upon for what Coccinelle generates and how one writes patches without
usage of SmPL. Even difference in space format could make working on a proof of
equivalence difficult.

The Coccinelle engine required a series of minor changes in order to provide a
conscice series of patches when using \#ifdefs. To provide a proof of equivalence
two identical trees can be used, one would have the SmPL rules applied while
the other would have a series of legacy patches applied. Proof of equivalence
is provided once the difference between these two trees yields no delta. This
legacy patch series to SmPL equivalence proof model was used to replace series
of patches within the backports project and it also revealed a little surprise:
the SmPL tree had more code changes, but the changes consisted of propagating the
expressed backported collateral evolution onto more drivers than previously
supported. It should be expected that generalizing a backport with SmPL
then would enable extending that backport onto more drivers. Once a driver
has the sum of all required collateral evolutions for one kernel addressed
it should be possible to enable that driver for usage on that kernel.
Should all required collateral evolutions to support one kernel be addressed
with SmPL then integrating one new device driver into the framework would
yield a complete automatic backport of that driver for the target kernel.

\subsection{Gains of using Coccinelle for backporting}

Coccinelle revealed inconsistencies on the threaded IRQ backport,
the new struct compat\_thread\_irq was pegged on different data
structures on different drivers. Its expected that Coccinelle
can help build more consistent and precise backports then.

Trying to backport threaded IRQ support with Coccinelle revealed
the series of patches carried to support threaded IRQ support
actually provided backporting for two collateral evolutions. The
backports project benefits from splitting up collateral evolutions
backported as it would in turn make it easier to backport the same
collateral evolution for other device drivers.

For a few patch series the amount of time to generate a backported release was
reduced, even though it automatically backported for example the threaded IRQ
collateral evolution to 10 new device drivers. The reason for the speed
increase of using Coccinelle over a series of patches consists of the
multithreaded nature of how one can apply rules with Coccinelle, while a series
of patches require linear synchronous application. Not all series of legacy
patches replaced with SmPL rules yielded shorter run times, but for some
series this certainly was the case.

\subsection{Scaling usage of Coccinelle to backport}

Backport releases are generated and compile tested on a big iron server donated
by HP, SUSE and Linux Foundation, the machine consists of 32 cores and 236 GiB
RAM.  The project uses a /pub/mem tmpfs, has linux.git, linux-stable.git,
backports, and all code it generates in RAM. Each release is test compiled
against every supported kernel, a Python script ckmake, is used to test
compilation of each release against all supported kernels.  Compilation tests
are all performed in RAM.

At first all SmPL patches were concatenated together and Coccinelle was run
only once. On uniprocessor runs / unparalleled environments this worked best.
Coccinelle has a script to help split up work for the engine, on each iteration
you speficy the max number of CPUs available, and the index which corresponds
to the current run, the rule and target. If you have 4 CPUs you spawn
Coccinelle 4 times and the only argument for each run that would change is the
index. The following script was typically used to help make use of Coccinelle's
parallelism:

\begin{lstlisting}[language=bash]
#!/bin/bash
# By Kees Cook
# http://comments.gmane.org/gmane.comp.version-control.coccinelle/680
set -e
MAX=$(getconf _NPROCESSORS_ONLN)
dir=$(mktemp -d)
for i in $(seq 0 $(( MAX - 1 )) ); do
    spatch -max $MAX -index $i -very_quiet "$@" > $dir/$i.out &
done
wait
cat $dir/*.out
rm -f $dir/*.out
rmdir $dir
\end{lstlisting}

\subsection{Coccinelle Parallelizing improvements}

The backports project ended up fine tuning usage of Coccinelle
with sensible arguments and managing Coccinelle's current parallelism
support using Python. The script was generalized for regular
kernel development use and now merged on Coccinelle development
git tree. Usage:

\begin{lstlisting}[language=bash]
pycocci foo.cocci drivers/
\end{lstlisting}

It does what you'd expect, it uses --in-place, its parallelized, and another
series sensible arguments are used if you want to apply rules broadly. In
the future Coccinelle will be modified to use Ocaml Parmap library
for integrated parallelism support. In order to help make Coccinelle
run faster software index utilities can be used, glimpse is one option,
another is idutils. Glimpse is unfortunately not FOSS. By default
Coccinelle relies on its own grep implementation to search for tokens.

\subsection{Helping Coccinelle find things faster: needle in the haystack}

At times it helps to express rules simply to locate objects
you wish to modify. A rule used as a needle in a haystack
search query does not need to contain changes, it would
simply contain elements describing the type of object you
want to consider modifying. For instance if you want to
modify a PCI driver you might use this first rule and
then depend on it for subsequent rules.

\begin{lstlisting}[language=diff]
@ module_pci @
declarer name MODULE_DEVICE_TABLE;
identifier pci_ids;
@@

MODULE_DEVICE_TABLE(pci, pci_ids);
\end{lstlisting}

Other rules can now depend on module\_pci and you'd be
sure to be only applying the following SmPL rules to
PCI drivers.

\begin{lstlisting}[language=diff]
@ simple_dev_pm depends on module_pci @
identifier ops, pci_suspend, pci_resume;
declarer name SIMPLE_DEV_PM_OPS;
declarer name compat_pci_suspend;
declarer name compat_pci_resume;
@@
+compat_pci_suspend(pci_suspend);
+compat_pci_resume(pci_resume);
SIMPLE_DEV_PM_OPS(ops, pci_suspend, pci_resume);
\end{lstlisting}

This last rule then adds a postfix for the backport respective callbacks.

\begin{lstlisting}[language=diff]
@@
identifier backport_driver;
expression pm_ops;
fresh identifier backports_pci_suspend = simple_dev_pm.pci_suspend ## "_compat";
fresh identifier backports_pci_resume = simple_dev_pm.pci_resume ## "_compat";
@@

struct pci_driver backport_driver = {
+#if (LINUX_VERSION_CODE >= KERNEL_VERSION(2,6,29))
	.driver.pm  = pm_ops,
+#elif defined(CONFIG_PM_SLEEP)
+	.suspend	= backports_pci_suspend,
+	.resume 	= backports_pci_resume,
+#endif
};
\end{lstlisting}

\subsection{Coccinelle is precise}

One of the benefits of Coccinelle is that its less prone to errors,
experience has shown manually backporting complex collateral evolutions
is error prone. In this section we'll provide an example of how a rule
will only modify the data structure specified, even if similar data
structures exist with similar member names.

The code example below can be obtained and tested at:

\url{https://github.com/mcgrof/netdev-ops.git}

Usage:

\begin{lstlisting}[language=bash]
make test1
git checkout -f
make test2
\end{lstlisting}

Consider the following piece of code:

\begin{lstlisting}[language=C]
#include

struct net_device_ops {
};

struct net_device {
	struct net_device_ops *netdev_ops;
};

struct bubble_ops {
};

struct bubbles {
	struct bubble_ops *netdev_ops;
};

static struct net_device_ops my_netdev_ops = {
};

static struct bubble_ops my_bubble_ops = {
};

static struct parent {
	struct net_device *dev;
	int b;
};

static struct parent_usb {
	struct net_device *net;
	int b;
};

int main(void)
{
	struct parent *p = malloc(sizeof(struct parent));
	struct parent_usb *p_usb = malloc(sizeof(struct parent));
	struct net_device *dev = malloc(sizeof(struct net_device));
	struct bubbles *bubble = malloc(sizeof(struct bubbles));

	dev->netdev_ops = &my_netdev_ops;
	bubble->netdev_ops = &my_bubble_ops;

	free(dev);
	free(bubble);
	free(p);
	free(p_usb);

	p->dev = dev;
	p->dev->netdev_ops = &my_netdev_ops;
	p_usb->net->netdev_ops = &my_netdev_ops;

	return 0;
}
\end{lstlisting}

The following SmPL grammar rule will modify both struct net\_device
and struct bubbles.

\begin{lstlisting}[language=diff]
@@
expression dev;
expression ops;
@@
-dev->netdev_ops = ops;
+netdev_attach_ops(dev, ops);
\end{lstlisting}

If one wanted to be a bit more precise and careful one
could be explicit about the data structure that
one wishes to modify:

\begin{lstlisting}[language=diff]
@@
struct net_device *dev;
struct net_device_ops ops;
@@
-dev->netdev_ops = &ops;
+netdev_attach_ops(dev, &ops);
\end{lstlisting}

This will only make changes to struct net\_device.

\subsection{Conclusions}

The backports project removed support for kernels older than 3.0, even though
it backports over 800 device drivers it currently only uses Coccinelle for 5
collateral evolutions. Using Coccinelle on small patches is possible but its
simply overkill. The ideal use case for Coccinelle when backporting is actually
for the harder collateral evolutions that need to be backported.

A general practice must be put in place to break down every patch into separate
atomic pieces.  Each atomic piece backported should have a respective
documented commit upstream. All changes dealing with the same commit upstream
should be grouped together, each set then represent the backport of one
collateral evolution. The more fine grained patches are broken down into the
easier it will be to transform a backport into SmPL.  This begs the question if
using SmPL for forward collateral evolutions might enable using the same SmPL
grammer but inversed to backport the same collateral evolution in the future.
Its purely theoretical, but it may be possible and such possibility would
take backporting automation to yet another new level.

The backports project has a series of best practices learned throughout the
years, for instance its easier to backport using static inlines and helpers.
If static inlines were already used upstream for setting / updating of certain
data structures the backports project would only have to provide the equivalent
definition for the call for older kernels. Would kernel maintainers be
comfortable in making changes upstream that would help automatically backport
features / device drivers / subsystems? The first set of changes submitted
upstream to help with making backporting easier was submitted and now merged.

\subsection{Acknowledgements}

A lot of the work put forward on backports with Coccinelle integration
was made possible thanks to funding by INRIA and IRILL.

\end{document}
