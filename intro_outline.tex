\section{Introduction}

Linux is an open-source operating system kernel that has been under
development since 1991.  Its reliability, customizability, and low cost has
made it a popular choice for an operating system kernel across the
computing spectrum, from smartphones and tablets based on the Android
distribution, to desktops based on popular distributions such as Debian and
Ubuntu, to supercomputers based on Enterprise Linux releases.  The Linux
kernel is evolving rapidly, with a major release roughly every 2.5 months.
Between the recent major releases v3.11 and v3.16 (September 2013 - August
2014), on average, 8,700 lines of code were added, 3,880 lines of code
were removed, and 1,900 lines of code were modified {\em every
  day}.\footnote{https://github.com/gregkh/kernel-development/blob/
  1b17a1f21f2b0b871419f29947b2960cb36cb00b/kernel-development.pdf?raw=true}
This rapid rate of change combined with frequent releases allows the Linux
kernel to keep up to date with bug fixes, new functionalities, and new
services, such as new CPU architectures, device drivers and filesystems.

\subsection{The dilemma for silicon manufacturers}

While the rapid release cycle of Linux has many benefits, it can be
problematic for certain classes of users.  Some system integrators may have
invested heavily in testing a specific release and may wish to avoid
regressions due to a kernel upgrade.  Upgrading a kernel may also require
experience to understand what features to enable, disable, or tune to meet
existing deployment criteria. In the worst case, some systems may rely on
components that have not yet been merged into the mainline Linux kernel.
Such a dependence may make it impossible to upgrade the kernel without
cooperation from the component vendor or a slew of partners that need to
collaborate on developing a new productized image for a system.  As an
example, development for 802.11n AR9003 chipset support on the upstream
ath9k device driver started on March 20, 2010 with an early version of the
silicon, at which point the most recent major release of the Linux kernel
was v2.6.32.  One of the first products to ship with this driver was the Google
Chrome CR48, using ChromeOS, which started selling in retail in May 2011.
The latest kernel release at this point was v2.6.38, but ChromeOS was still
based on the v2.6.32 kernel, the release it was originally developed for.

The reluctance of users in many markets to keep up to date with the latest
kernel poses a dilemma for silicon manufacturers, who need to make
available device drivers so that their devices can be used on Linux
systems.  One approach is to develop drivers explicitly for the kernel
releases that their clients are currently using.  However, even clients who
value stability may eventually need to modernize.  Doing so then incurs the
burden of a full rewrite or port of each driver to the newly adopted kernel
release.  While the device itself may be unchanged, the kernel evolves
frequently with new internal APIs to improve the performance, robustness or
flexibility of the code.  Pervasive {\em collateral evolutions} are then
needed to update the driver with respect to these changes
\cite{Padioleau:eurosys06}.  And even once the driver is successfully
ported to a more modern kernel, the problem is not really solved.  The
result will only be usable by those who are currently using the same
kernel release.

An alternative to targeting a device driver to a specific kernel version is
\textit{upstream-first} development, in which code is initially developed
only for the latest major kernel release, and then is submitted for
inclusion {\em upstream}, {\em i.e.}, into the Linux git repository
maintained by Linus Torvalds, allowing the device driver to be included in
the coming major release.  While achieving inclusion upstream can be a
challenge for silicon manufacturers, due to the strict coding guidelines of
the Linux kernel, once it is achieved, the developers at the silicon
company that upstreamed the device driver can then benefit from help from
the Linux community in reviewing changes to the device driver as the Linux
kernel evolves \cite{KH:api}.  The silicon manufacturer needs to contribute
only one complete version of the code; since it is upstream it will then be
part of all future kernel releases and all future Linux distributions. As
time goes by the silicon manufacturers may wish to turn their attention
towards newer silicon.  At this time, engineers can then orphan maintenance
for older device drivers, allowing any community developer to take their
maintenance on.

\subsection{Why and how Linux is backported}

The upstream-first model makes device drivers available for users of future
kernels, but leaves out those who must remain with older releases.  If a
device driver is only supported upstream, a Linux distribution or system
integrator has no option but to backport that device driver down to the
kernel release of interest.  Backporting strategies have typically
consisted of augmenting each affected file with \#ifdefs that handle what
is required for each kernel release on each target file. As each file is
augmented individually, there is no code sharing, even within a single
Linux distribution's codebase. Files also become harder to read, as the
original code is interspersed with new code flows required to support each
kernel release.

Since 2007, the Linux kernel backports project has promoted an alternative
backport strategy, with the goal of maximizing code sharing and minimizing
disruption to the individual driver source files, and with the goal of
enabling upstream-first development of new drivers and features by making
backports available to everyone, regardless of the Linux distribution used.
The main innovation was to move the changes required to backport each
driver out of the individual driver files and into a {\em compat} library,
providing a set of helper functions.  Indeed, typically, for a given class
of devices, the drivers use a similar set of API functions and coding
strategies, and these functions and coding strategies can all be backported
in the same way.  Rather than distributing the handling for each older
release in every driver file, all of these variations are encapsulated into
a compat library function.  These functions can be declared as static
inlines when performance is needed, or as external functions to limit the
increase in code size.  This approach was used in the ath9k support for
ChromeOS noted above. The ath9k device driver was extended with AR9003
family chipset support upstream, and this support was incorporated as part
of the v2.6.38 release at the time of the release of the Google Chrome
CR48. Support for the AR9003 family of chipsets on ath9k on ChromeOS
however was provided and backported onto the ChromeOS v2.6.32 based kernel
using the Linux backports compat library.

%% The repetitive \#ifdefs and backport-specific code are typically replaced
%% by at most a single \#ifdef adding a single function call at each point
%% where the upstream behavior is not appropriate for older kernels.

The use of a compat library can dramatically reduce the amount of code
changes required to support a class of drivers.  Nevertheless, some changes
per driver are still required, amounting to {\em glue code}, to invoke the
compat library functions and to {\em e.g.}, modify type definitions, which
cannot be encapsulated into a function definition.  These changes must
initially be made manually in each supported file to create {\em
  patches},\footnote{A patch is a document indicating the lines of added
  and removed code, in a format generated by the Unix command {\tt diff}.
  A patch can be automatically applied to a file using the Unix command
  {\tt patch} \cite{diffmanual:02}.} which are made available to users, and
these patches need to be maintained as the kernel evolves.  Making these
changes and maintaining the resulting patches is tedious and error prone,
and limits the number of drivers that the backports project can support.  A
solution was thus needed to ease the process introducing this glue code.

\subsection{Our contribution}

In this paper, we report on a new methodology for backports adopted by the
Linux backports project that combines the use of a compat library with the
use of Coccinelle \cite{Padioleau:eurosys08}. Coccinelle is a program
matching and transformation tool for C code that has been specifically
designed to meet the needs and requirements of Linux developers.  Matches
and transformations are described in terms of an extension of the patch
notation with generic features, resulting in {\em semantic patches}.
Unlike patches, which are restricted to specific positions in specific
files, a Coccinelle semantic patch expresses a change in a generic way,
allowing it to be applied across an entire code base and to adapt to minor
changes in this code base, as the code base evolves over time.  In the
context of backports, we automate the integration of the glue code into
each supported driver using Coccinelle.  Now the glue code needs to be
specified only once, and then this specification can be applied
automatically to all supported drivers over multiple releases, thus further
reducing the amount of maintenance work to support a backport.

The rest of this paper is organized as follows.  Section \ref{background}
provides the background for this work, including the relevant aspects of
the Linux development models, the history of the backports project, and the
use of Coccinelle.  Section \ref{strategies} then presents a tour of
backporting strategies in more detail, based on a simple example. Next,
Section \ref{case} expands this tour to illustrate a more complex case
study, in which the changes required are determined by driver-specific
information.  Section \ref{correct} then highlights some correctness and
performance issues.  Finally, we briefly consider related work,
conclusions, and directions for future work.  This paper will put emphasis
on how each new backporting strategy has helped the backports project to
grow and scale, and ultimately how each of these strategies has helped to
shift the objective of the project away from simply backporting the Linux
kernel, towards trying to automate the backport process as much as
possible.

