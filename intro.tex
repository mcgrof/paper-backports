\section{Introduction}

\subsection{The Linux kernel development model}

The Linux kernel versioning scheme consists of a version, patchlevel,
sublevel, and extraversion, all named after each attributes defined
on the project top level Makefile. A \textbf{\textit{major kernel}} release
happens when the version changes or when the version remains the same but the
patchlevel is updated. A \textbf{\textit{major kernel release}} always has an
implicit sublevel of 0. A \textbf{\textit{stable release}} follows on top of a
\textbf{\textit{major kernel release}}, both the version and patch level remain
the same but the sublevel updates. A \textbf{\textit{release candidate}}
consists of a version, patchlevel, an implicit sublevel of 0, and an
extraversion prefixed with "rc" and a digit representing the release
candidate number, starting from 1. For example the v3.16-rc1 release has a
version 3, sublevel 16, an implicit sublevel 0, and rc1 extraversion.

Development for Linux happens through a set of distributed development
trees, each development tree exclusively handles a set of components known
as \textbf{\textit{subsystems}}, the list of components each tree is
responsible for is documented on the project MAINTAINERS\footnote{https://git.kernel.org/cgit/linux/kernel/git/torvalds/linux.git/tree/MAINTAINERS} file.
Each development tree carries a delta for its components on top of the
latest known \textbf{\textit{major kernel release}} release. Each
\textbf{\textit{subsystem}} has a respective
\textbf{\textit{subsystems maintainer}} or group of
\textbf{\textit{subsystems maintainers}}. All new kernel features and
drivers are queued on \textbf{\textit{subsystems development trees}} and in order
to help make a release each \textbf{\textit{subsystem maintainer(s)}}
eventually sends a pull request of their \textbf{\textit{subsystem development tree}}
to Linus Torvalds during the \textbf{\textit{merge window}}.
The \textbf{\textit{merge window}} begins
as soon as Linus Torvalds release the next \textbf{\textit{major kernel release}}
of Linux, the \textbf{\textit{merge window}} culminates when Linus Torvalds
release the first \textbf{\textit{release candidate}} of a
new \textbf{\textit{major kernel release}}; this entire process typically
lasts about 1-2 weeks. For instance, the v3.16
\textbf{\textit{merge window}} happened in between the release of
v3.15 (June 8, 2014) and v3.16-rc1 (June 15, 2014).  A period known as the
\textbf{\textit{release candidate evaluation}} period
or \textbf{\textit{rc cycle}} follows the release of the first
\textbf{\textit{release candidate}} after the \textbf{\textit{merge window}}.
A \textbf{\textit{major kernel release}} may go through 5-9
\textbf{\textit{release candidates}}.

The purpose of the \textbf{\textit{release candidate evaluation}} period ideally
is to only find and fix regressions for new code merged since the last
\textbf{\textit{major kernel release}}. For example the
v3.16 \textbf{\textit{major kernel release}} had 7 \textbf{\textit{release candidates}},
v3.16-rc1 - v3.16-rc7, prior to Linus having made the final v3.16 \textbf{\textit{major kernel release}}.
\textbf{\textit{Subsystem maintainer(s)} queue regression fixes for the
current \textbf{\textit{release candidate}} on a \textbf{\textit{subsystem stable tree}},
used only to send pull requests to Linus during the
\textbf{\textit{release candidates evaluation}} period.
A new driver by definition could not have been present on prior major release, in this case v3.15, and
as such an exception is made to allow new device drivers during the
\textbf{\textit{release candidate evaluation}} period. A
\textbf{\textit{major kernel release}} is followed by
\textbf{\textit{stable releases}} for it, for instance v3.16 was followed
by v3.16.1, v3.16.2 and will continue to have releases until the
kernel is declared as unmaintained. \textbf{\textit{Stable releases}} only
contain critical bug fixes, they may never contain new features or drivers.

The linux-next\footnote{git://git.kernel.org/pub/scm/linux/kernel/git/next/linux-next.git}
tree is used to mimic the \textbf{\textit{merge window}}
on daily basis by pulling each \textbf{\textit{subsystems tree}} through
a set of scripts. Every day the tree is reset to the latest
\textbf{\textit{major kernel release}} and then each
\textbf{\textit{subsystems tree}} pulled, after all trees are pulled
the tree as tagged to reflect reflect the calendar date. The linux-next
tree could be used to track the latest development efforts on all
subsystems on a daily basis.


\subsection{Current pace of Linux kernel development}

The Linux kernel has a rapid pace of development
\footnote{https://github.com/gregkh/kernel-development/blob/1b17a1f21f2b0b871419f29947b2960cb36cb00b/kernel-development.pdf?raw=true}:
in between the releases v3.11.0 - v3.16.0 (September 2013 - August 2014) we
have record of 3,483 developers and 439 companies contributing to the project.
During this period 8,700 lines of code were added, 3,880 lines of code removed,
and 1,900 lines of code modified on average every day. This accounts for
roughly about 7.6 changes per hour. The v3.16 release in particular had 9.5
changes per hour. The Linux kernel currently has a release cycle of about one
release every 2 1/2 months. With such a rapid pace of development users, Linux
distributions and system integrators are found having to cope with either
upgrading their kernels regularly or providing a means to integrate features /
drivers from future kernels into their existing kernels. There are a slew of
reasons why some users may not want to upgrade their kernels as often as
kernels are released. Some system integrators may have invested heavily in
testing a specific release and may wish to avoid regressions or invalidation of
tests due to a kernel upgrade.  Upgrading a kernel may also at times require
experience to understand what features to enable / disable or tune to meet
existing deployment criteria. In the worst case some systems may rely on
components not yet merged upstream on Linux and a kernel upgrade may at times
be impossible without effort of a vendor or a slew of partners collaborating
towards a new productized image for a system.

The Linux kernel follows a loose timeline model for releases. There are known
trackers to produce an evaluation and best guess estimate of the next
release\footnote{http://phb-crystal-ball.org/} based on evaluation of history
of prior releases, these trackers are however currently only use very simply
heuristics and should only be used for estimation purposes only. Based on the
release history in between v3.11 - v3.16  (September 2013 - August 2014) we
know the Linux kernel had roughly about one release every 2 1/2
months\footnote{https://github.com/gregkh/kernel-development/blob/1b17a1f21f2b0b871419f29947b2960cb36cb00b/kernel-development.pdf?raw=true}.
During this time of 11 months period there were 5 kernel releases:
v3.11 - v3.16. How Linux distributions, users, system integrators end up
picking up new a \textbf{\textit{major kernel releases}} for their use varies
widely but since new device drivers and features can only come from
\textbf{\textit{major kernel releases}} the only way to get these features
unless they are backported is to do a full kernel upgrade.

\subsection{Benefits and issues with upstream development}

When features or device drivers are written specifically for older
\textbf{\textit{major kernel releases}} they incur an additional burden of
either a full rewrite or port when its that device driver or features is
desirable on a new \textbf{\textit{major kernel release}}.  The amount of work
required to address a forward port for a device driver or feature consists of
addressing all collateral evolutions for that feature / device driver for all
kernel releases it has no forward port for yet. For example, a networking
device driver written only for the v3.2 kernel will have to have addressed all
collateral evolutions in between v3.12 - v3.16 in order to be supported on the
v3.16 release of the kernel. A device driver or feature written only for one
target \textbf{\textit{major kernel release}} will also only be useful for
those users / Linux distributions / system integrators who are currently using
the same \textbf{\textit{major kernel release}}.  When code is developed
targeting the latest \textbf{\textit{major kernel release}} and submitted for
inclusion upstream, a practice known as \textbf{\textit{working upstream}} or
\textbf{\textit{upstream first model}}, it will gain from being advanced by the
community as the kernel evolves, it will always be compliant with the latest
\textbf{\textit{major kernel release}}. Not all users will be using the latest
\textbf{\textit{major kernel release}} however, and this poses a problem of
where to dedicate resources and attention to when doing development.

There are two differnt types of problems depending on which development model
is embraced: to target an old kernel release or to work with with an
\textbf{\textit{upstream first model}}. The Linux community prefers you to base
your work on top of the latest and greatest, for you to \textbf{\textit{work
upstream}}, given that if you are working on an old release of the kernel
chances your code will be deprecated fast. If you are working on code based on
an old release the users of your code remain constrained to one specific kernel
release. If you \textbf{\textit{work upstream}} and you submit your code as
early as possible all Linux distributions will eventually get support for your
driver / features. At least for silicon manufacturers though, at the time a
device driver is being written they need to provide support for the current set
of \textbf{\textit{major kernel releases}} that are currently supported. This
and the fact that Linux kernel development follows strict coding guidelines and
best practices may at time deter silicon manufacturers from prioritizing doing
Linux development with an \textbf{\textit{upstream first model}}.

\subsection{Why and how Linux is backported}

If a device driver or feature is supported upstream on the latest
\textbf{\textit{major kernel release}} but not on older kernel
releases a Linux distribution or system integrator has no option
but to do work to backport that device driver or feature down to
the target kernel release of interest. If two Linux distributions or
system integrators require the same backport work for one device driver
or feature that code has historically not been generalized or shared
due to a series of reasons: different timelines, priorities,
different work flows. This also means that historically Linux backporting
work has generally not been shared between interested parties.

Backporting strategies have typically consisted of wrapping kernel
code with C \#ifdefs handling what is required for each kernel release on each
target file. This means backported code for each component has typically not
been shared even within each Linux distribution's or system integrator's
codebase. This also means that code that gets backported becomes harder to read
given the new code flows required to support each kernel release.

\subsection{Generalizing backporting Linux}

A solution to all these problems was desirable in order to help promote an
\textbf{\textit{upstream first}} development model, it was imperative that the
solution remain Linux distribution agnostic. What was required was a mechanism
by which all development can be focused on the latest \textbf{\textit{major kernel
release}} and a backport solution provided that would easily allow usage of
that code on a series of older \textbf{\textit{major kernel releases}} but at
the same time also provide as much code sharing, make code easy to read and
maintain. Sharing code for a backport through a series of helpers can help by
reducing the amount of \#ifdefs on code, this practice can help split up,
categorize and most importantly identify required required backport work by
documenting the required specific collateral evolutions that are being
addressed on each backport. This practice helps identify commonly backported
collateral evolutions, this is important for both documentation purposes but
also helps with identifying code that should probably be considered for a
helper.

Sharing code helps backporting collateral evolutions by requiring less work,
target code however needs to be transformed to use a new set of shared helpers.
Using a set of helpers allows to reduce the amount of changes required on
upstream code to support different kernel releases.  This process still
requires manual intervention and as such can be error prone.  Coccinelle is an
engine that was developed to evolve the Linux kernel through a grammar, SmPL.
Use of Coccinelle was evaluated to help perform the transformation required by
a shared set of helpers. If possible it would mean gramatically expressing
the backport of a collateral evolution. If a collateral evolution backport
could be formalized through SmPL it would mean having to express the backport
only once, reducing the mount of code supported for a backport.

\subsection{Overview}

This paper will review how the above problems gave rise to the Linux backports
project \footnote{https://backports.wiki.kernel.org}, describe in detail with
examples how backporting has historically been done in the industry, how the
project evolved the practice to use a shared backports module and set of
headers files to reduce the amount of delta for backported components, and
finally how and why it has integrated support for use the Coccinelle engine in
order to help the project. To help with this the paper includes a specific case
study example of a complex collateral evolution the project had to backport and
will review in detail that backport using each different approach discussed.
The first backport approach reviewed will be the strategy used by most of the
industry using \#ifdefs on target code, it will then explain how that backport
can be generalized through a shared backport module, and finally how it can be
further optimizing by integrating Coccinelle SmPL. There are a series of minor
challenges that will be discussed faced with using Coccinelle but these were
all related to parrallelism and pretty printing, and growing confidence with
Coccinelle by proving equivalence of a backport through a patch series Vs using
Coccinelle SmPL. The different use case of Coccinelle for backporting will be
explained and how its use case contrasts what was expected for general Linux
kernel development. This paper will put emphasis on the idea of of how each new
strategy embraced over the years has helped the project grow and scale and
ultimately how each of these these new strategies have helped shape the project
objective shift from just backporting the Linux kernel towards trying automate
the proces of Linux kernel backporting as much as possible.
